\documentclass{article}
\usepackage[utf8]{inputenc}
\usepackage{amsmath}
\usepackage{amsfonts} % enable the use of \mathbb
\usepackage{physics}



\title{Chapter 1}
\author{Yu Lu}

\begin{document}
% define commands for special state vectors, p means positive, n means negative
\newcommand{\pz}{\ket{\vb{+z}}}
\newcommand{\nz}{\ket{\vb{-z}}}
\newcommand{\px}{\ket{\vb{+x}}}
\newcommand{\nx}{\ket{\vb{-x}}}
\newcommand{\py}{\ket{\vb{+z}}}
\newcommand{\ny}{\ket{\vb{-y}}}

% writes expression of |+x> etc. in |z> basis
\newcommand{\pxexpr}{ \frac{1}{\sqrt{2}} \pz + \frac{1}{\sqrt{2}} \nz}
\newcommand{\nxexpr}{ \frac{1}{\sqrt{2}} \pz - \frac{1}{\sqrt{2}} \nz}
\newcommand{\pyexpr}{ \frac{1}{\sqrt{2}} \pz + \frac{i}{\sqrt{2}} \nz}
\newcommand{\nyexpr}{ \frac{1}{\sqrt{2}} \pz - \frac{i}{\sqrt{2}} \nz}
\maketitle
The two tables below shows correspondence of question index in 2nd Edition of the textbook and that in 1st Edition of the textbook.
\begin{table}[h!]
    \begin{center}
        \begin{tabular}{l|c|c|c|c|c|c|c}
            \hline
            2nd Edition
            &1.1
            &1.2
            &1.3
            &1.4
            &1.5
            &1.6
            &1.7
            \\ \hline
            1st Edition
            &1.1
            &1.2
            &1.3
            &1.4
            &1.5
            &1.6
            &1.7
            \\ \hline
        \end{tabular}
    \end{center}
\end{table}
\begin{table}[h!]
    \begin{center}
        \begin{tabular}{l|c|c|c|c|c|c|c|c}
            \hline
            2nd Edition
            &1.8
            &1.9
            &1.10
            &1.11
            &1.12
            &1.13
            &1.14
            &1.15
            \\ \hline
            1st Edition
            &1.8
            &
            &
            &
            &
            &1.9
            &
            &
            \\ \hline
        \end{tabular}
    \end{center}
\end{table}
\section*{Problem 1.9}
Skipped
\section*{Problem 1.10}

According to Problem 1.3,
\begin{equation} 
    \ket{\vb{+n}} = \cos{\frac{\theta}{2}} \ket{\vb{+z}} + e^{i\phi} \sin{\frac{\theta}{2}} \ket{\vb{-z}}
    \label{P1.10-orientation vector}
\end{equation}
Therefore,
\begin{equation} \nonumber
    \begin{cases}
        \cos{\frac{\theta}{2}} &= \frac{1}{2} \\
        e^{i\phi} \sin{\frac{\theta}{2}} &= \frac{\sqrt{3}}{2} i
    \end{cases}
\end{equation}
\begin{equation*}
    \theta ={} \frac{2}{3} \pi, \quad 
    \phi ={} \frac{\pi}{2}
\end{equation*}
\begin{equation} \nonumber
    \boxed{
    \begin{split}
        \vb{n} &={} 
    <\sin{\theta}\cos{\phi}, \sin{\theta}\sin{\phi},\cos{\theta}> \\
    &={}
    <0,\frac{\sqrt{3}}{2}, -\frac{1}{2}>
    \end{split}}
\end{equation}
Using $P(\lambda_i)=\abs{\braket{\lambda_i}{\psi}}^2$ and the condition $\ket{\psi}=\frac{1}{2}\ket{+\vb{z}}+\frac{\sqrt{3}i}{2}\ket{-\vb{z}}$,  
\begin{equation} \nonumber
    P(\vb{+z}) = \frac{1}{4}, \quad
    P(\vb{-z}) = \frac{3}{4}
\end{equation}
Using the definition of expectation value $\expval{A}=\sum_i{P(\lambda_i)\lambda_i}$,
\begin{equation*}
    \boxed{
    \begin{split}
        \expval{S_z} =&{} \frac{\hbar}{2} P(\vb{+z}) + (-\frac{\hbar}{2}) P(\vb{-z})\\
    =&{}  -\frac{\hbar}{4}
    \end{split}}
\end{equation*}
Considering $\ket{\vb{+x}} = \frac{1}{\sqrt{2}} \ket{\vb{+z}} + \frac{1}{\sqrt{2}} \ket{\vb{-z}}$ and following a similar procedure,

\begin{equation*}
    P(\vb{+x}) = \frac{1}{2}, \quad
    P(\vb{-x}) = \frac{1}{2}
\end{equation*}

\begin{equation*}
    \boxed{
    \expval{S_x}=0}
\end{equation*}
Considering $\ket{\vb{+y}} = \frac{1}{\sqrt{2}} \ket{\vb{+z}} + \frac{i}{\sqrt{2}} \ket{\vb{-z}}$ and following a similar procedure,

\begin{equation*}
    P(\vb{+y}) = \frac{2-\sqrt{3}}{4}, \quad
    P(\vb{-y}) = \frac{2+\sqrt{3}}{4}
\end{equation*}

\begin{equation*}
    \boxed{
    \expval{S_y} =-\frac{\sqrt{3}}{4} \hbar}
\end{equation*}

\section*{Problem 1.11}
We can show that these two phases only differ by an overall phase factor by calculating their relative phase difference.\\
Denote $\ket{\psi_1}=\frac{1}{2}e^{i\delta_+ }\ket{\vb{+z}}+\frac{\sqrt{3}}{2}e^{i\delta_-}\ket{\vb{-z}}$ and $\ket{\psi_2}=\frac{1}{2}e^{i\gamma_+ }\ket{\vb{+z}}+\frac{\sqrt{3}}{2}e^{i\gamma_-}\ket{\vb{-z}}$,
\[
    \delta_+ = 0, \quad \delta_- = \frac{\pi}{2}
    \Rightarrow \delta = -\frac{\pi}{2}  
\]
\[
    \gamma_+ = -\frac{\pi}{2}, \quad \gamma_- = 0
    \Rightarrow \gamma = -\frac{\pi}{2}
\]
The relative phase difference $\delta=\gamma$, thus the two states are equivalent, indicating
\begin{equation*}
    \boxed{
        \expval{S_z} = -\frac{\hbar}{4}, \quad
        \expval{S_x} = 0, \quad
        \expval{S_y} = -\frac{\sqrt{3}\hbar}{4}
    }
\end{equation*}
\section*{Problem 1.12}
Following \eqref{P1.10-orientation vector},
\[
    \theta= \frac{3\pi}{2}, \quad
    \phi = 0
\]
thus
\[
    \vb{n}=
    <\frac{\sqrt{3}}{2},0,-\frac{1}{2}>
\]
As $\ket{\psi}=\frac{1}{2}\pz + \frac{\sqrt{3}}{2} \nz$,
\[
    P(+z) = \frac{1}{4}, \quad P(-z) = \frac{3}{4}
\] 
\[
    \boxed{
    \expval{S_z} = -\frac{\hbar}{4}}
\]
As $\px = \pxexpr $,
\[
    P(+x) = \frac{2+\sqrt{3}}{4}, \quad
    P(-x) = \frac{2-\sqrt{3}}{4}
\]
\[
    \boxed{
    \expval{S_x} = \frac{\sqrt{3}}{4}\hbar}
\]
As $\py = \pyexpr $,
\[
    P(+y) = \frac{1}{2}, \quad
    P(+y) = \frac{1}{2}
\]
\[
    \boxed{
        \expval{S_y} = 0
    }
\]

\section*{Problem 1.14}
Because $P(\lambda_i) = \abs{\braket{\lambda_i}{\psi}}^2$ and in light of the overall phase irrelevency, we can write $\ket{\psi}$ as 
\[
    \ket{\psi} = e^{i\theta}
    (\frac{3}{5}\pz + e^{i\phi} \frac{4}{5} \nz)
\]
As $\px = \pxexpr$, the probability of measuring $+x$ would be 
\[
    \begin{split}
        P(+x)
        =& \abs{\frac{3}{5\sqrt{2}}+e^{i\phi}\frac{4}{5\sqrt{2}}}^2 \\
        =& (\frac{3}{5\sqrt{2}}+e^{i\phi}\frac{4}{5\sqrt{2}}) (\frac{3}{5\sqrt{2}}+e^{-i\phi}\frac{4}{5\sqrt{2}}) \\
        =& \frac{1}{50} (25+12e^{i\phi}+ 12e^{-i\phi}) \\
        =& \frac{1}{2} + \frac{24}{50} \cos{\phi}
    \end{split}
\]
It is given that $P(+x)=\frac{1}{2}$, so 
\[
    \cos{\phi} = 0 \Rightarrow 
    \phi = \frac{\pi}{2} \Rightarrow
    e^{i\phi} = i
\]
Therefore,
\[
    \boxed{
        \ket{\psi} = 
        e^{i\theta} (\frac{3}{5} \pz + \frac{4i}{5} \nz)
        \quad \theta \in \mathbb{R}}
\]

\section*{Problem 1.15}
Similar to Problem 1.14, we can express
\[
    \ket{\psi}= e^{i\theta}
    ()\frac{3}{\sqrt{10}} \pz + \frac{1}{\sqrt{10}}e^{i\phi}\nz)
\]
As $\py = \pyexpr $, the probability of measuring $+y$ would be 
\[
    \begin{split}
        P(+y)
        &= \abs{\frac{3}{\sqrt{20}}+\frac{1}{\sqrt{20}}ie^{-i\phi}}^2 \\
        &= \frac{1}{20} (9+1-3i(e^{i\phi}-e^{-i\phi})) \\
        &= \frac{1}{2} + \frac{3}{10} \sin{\phi}
    \end{split}
\]
It is given that $P(+y) = \frac{1}{2}$, so
\[
    \sin{\phi} = -1 \Rightarrow
    \phi = -\frac{\pi}{2} \Rightarrow
    e^{i\phi} = -i
\]
Therefore,
\[  
    \boxed{
    \ket{\psi} = 
    e^{i\theta} (\frac{3}{\sqrt{10}} \pz - \frac{i}{\sqrt{10}} \nz)
    \quad \theta \in \mathbb{R}
    } 
\]
As $\px = \pxexpr$,
\[
    \boxed{
    P(+x) = 
    \abs{\frac{3}{\sqrt{20}}-\frac{1}{\sqrt{20}}i}^2
    = \frac{1}{2}
    }
\]
\end{document}